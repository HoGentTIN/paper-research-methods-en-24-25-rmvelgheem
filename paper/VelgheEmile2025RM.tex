%==============================================================================
% Template research proposal bachelor thesis
%==============================================================================

\documentclass[english]{hogent-article}

% Specify bibliography file
\addbibresource{references.bib}

% Information about the study programme, course, assignment
\studyprogramme{Bachelor of applied information technology}
\course{Research Methods}
\assignmenttype{Paper Research Methods: research proposal}
\academicyear{2024-2025}

\title{Can personal OS-integrated artificial intelligence revolutionize task management for people with ADHD compared to current cloud-based AI software?}

\author{Emile Velghe}
\email{emile.velghe@student.hogent.be}

\projectrepo{https://github.com/HoGentTIN/paper-research-methods-en-24-25-rmvelgheem.git}

% Within which specialization from the final year of the study programme
% is this research situated? Choose from this list:
%
% - Mobile \& Enterprise development
% - AI \& Data Engineering
% - Functional \& Business Analysis
% - System \& Network Administrator
% - Mainframe Expert
% - If the research does not fit within one of these domains, specify it
%   yourself
%
\specialisation{AI \& Data Engineering}
\keywords{ADHD, AI, task management, OS, Personal Intelligence}

\begin{document}

\begin{abstract}
  People with ADHD (Attention Deficit Hyperactivity Disorder) often struggle organizing their daily obligations and prioritizing important tasks because of the lack of dopamine in their brain. To help with this, calendar apps started integrating and using AI to automate and personally tailor people’s calendar to their lifestyle. While these applications promise many useful features and plenty of customization options, they often require a long and time-consuming setup process to be truly personal to every user. For people with ADHD, this process can sometimes be too overwhelming and time consuming, which leads to them not using the app after all. Apple's new AI software, however, could potentially solve this problem once and for all, as it no longer requires any sort of configuration before it can be used to it's full potential. This study investigates the difficulties people with ADHD experience when using cloud-based calendar apps and explores how new the artificial intelligence software could potentially be the solution to these problems. This is done by having a selected group of people with ADHD use existing apps as their daily calendar, and then compare their experiences to using the new, OS-integrated, software as their daily calendar. It is expected for these people to find the new software much more convenient, as it is already trained on personal data, giving it a truly tailored experience from the beginning. 
\end{abstract}

\tableofcontents

\bigskip

\paragraph{Remark: improvements to the original proposal}

\section{Introduction}%
\label{sec:Introduction}

One of the main difficulties people with ADHD experience is task and time management, as they often lose track of time very quickly, resulting in missed deadlines and forgotten appointments. For the past decade, developers started adding artificial intelligence and machine learning software to apps like calendars and reminders to automate task management. This way, the app can provide the user with custom smart suggestions. To make the experience even more personal, these apps typically require some sort of setup process or survey in which the user provides them with personal information. This way, the app can learn about the user and adapt to its habits. Although these apps are so called “smart” or “intelligent”, the user still has to manually enter their appointments or reminders as the app doesn't have access personal data such as locations or incoming texts. People with ADHD often don't have the patience or mental energy to go through this process and so, they end up abandoning the app. Apple's newly announced AI software however, could possibly be the solution to this problem. This new personal AI isn't cloud-based and works at kernel level, giving it instant access to all user data and allows it to take actions in apps across the device. Having a personal assistant in your pocket, helping you with planning and task management across multiple apps could possibly be the ultimate solution to the difficulties neurodivergent people experience in their day to day lives. This study describes why modern AI powered calendar apps are not optimized for people with ADHD and why new AI technology could possibly change that. The literature review will first investigate and describe the different kinds of AI software used today and what their limitations are. This software will then be compared to the newly developed AI technology to determine the main differences in functionality. Lastly, the study will determine the potential use cases and improvements this new software has in assisting neurodivergent people with planning and task management.


\section{Literature Review}%
\label{sec:literature review}

Major theories of ADHD have emphasized core deficits in the cognitive processes associated with attention and executive function (EF), which includes processes such as behavioral inhibition, working memory, and organization and planning skills ~\autocite{Molitor2017}.

Normally, people would just start using any kind of calendar either online or on paper when they have to schedule appointments or be reminded about important tasks. People with ADHD however, find it just as difficult to build and follow a schedule because that is just the way their impulsive brain works. They can never predict what they will be doing in the future. In the real early stages of software development, apps were designed specifically to help people in general with planning their days. It was basically just a visual of your calendar as you would normally have on paper, but now it was on your phone. These apps changed nothing to the general activity of planning. This is where artificial intelligence could normally offer some help, as it is designed to think like real humans. 

AI has the potential to improve treatment for individuals with ADHD. For example, AI-powered virtual assistant or chatbot could provide round-the-clock support, guidance, and scheduling to improve attention and executive functioning skills in managing ADHD. A similar technology has already been adapted for other mental health conditions, such as anxiety, depression, and stress. ~\autocite{Rahman2023}.

Artificial intelligence has improved significantly over the last five to ten years and has been implemented into a lot of different applications. One of these is the use of AI in day to day scheduling and task management apps. Popular apps such as: Todoist, Trello, and Microsoft To Do have already shown to be very helpful with this. These apps can assist people with ADHD by automating and simplifying the process of task organization and time management. They use AI to learn about a user’s habits and preferences and constantly adapt to this to provide even more accurate support. 

According to the research by Cho and Kim, LLMs effectively engaged patients empathetically and adaptively. However, it also identified limitations in personal interactions and understanding emotions to the depth that a human therapist would do ~\autocite{Berrezueta-Guzman2024}.

One of the biggest downsides about current models is the need for users to share personal information with their AI system to personalize responses to their own situation. In the early stages of AI development, the level of “smart” assistance AI powered applications offered, when used for the first time, was generally the same for every user, as they were trained by the same data sets. These datasets provide the AI software with one general image of how the average person schedules their day. This means that the app could now, for example, estimate how long it would take for a simple task to complete and change the schedule accordingly.

To change this, developers need information about real individuals and how they behave in a day to day life, but the real question is where to get this information. 
The only real personal data these apps, and other software applications, have access to, is basic personal information such as: age, sex, country of birth, things you like, and often your location as well. This is the only information separating assistance for one user to another. By using the app for an extended amount of time, it learns more about the user and its habits to provide more personal suggestions. However, the app still doesn't have access, and also won’t have access in the future,  to real personal data such as: texts, photos, notes or calls.

Thanks to AI, all kinds of personal data can be used to analyse, forecast and influence human behaviour, an opportunity that transforms such data, and the outcomes of their processing, into valuable commodities. In particular, AI enables automated decision-making even in domains that require complex choices, based on multiple factors and non-predefined criteria ~\autocite{Sartor2020}.

Therefore, apps that use AI, when just downloaded, try to collect as much information about its user as possible to compete with other apps by offering the best personal user experience. This can be done in multiple ways such as filling out a question form when setting up the app, or giving the app permission to different kinds of data across your device. This information is needed if the app wants to make good on its promises and give a good first impression. For many people, especially those with ADHD, the first impression when using any kind of app for the first couple of days is very important, as this usually determines whether or not they will be using it in the long term. This explains why paid applications usually come with a 7-day free trial. When downloading the app called ‘Motion’ for example, it first asks you to connect your current calendar. It then asks the user to enter some of their own tasks they have to complete and recommends entering at least five, so the app’s AI can respond more accurately. When you complete this setup, you are greeted with a pretty minimal interface that looks basically like any other scheduling interface. Next, a step by step user guide is offered and when you are finally ready to utilize the app, a pop-up asks you to pay over €300 yearly if you want to continue using it

This whole setup process and new interface the user has to get used to could potentially lead to people with ADHD to not use the app, as this takes a lot of mental effort. If a user manages to get through the first days or weeks while using the app as their personal assistance, they will notice more personal suggestions for their situation. This is the result of the AI tracking their behaviour across multiple parameters. It tracks for example your location, sleep schedule, notifications, screen time, content you watched or liked, and combines this data all together to get an idea of your personality. However, knowing some app is tracking every move you make, people started wondering what other data is being monitored and what happens to it once the software has processed it. Given the fact that nowadays, almost every AI software is cloud-based, this essentially means that your personal data is being sent across the internet to various servers and databases. This ultimately raises a lot of questions about privacy among users.

“While AI-powered apps are used to analyze individual’s daily activities and movement patterns to detect signs of ADHD and keep them organized, there is a potential concern about individual’s privacy and identification. These apps may collect sensitive personal information, such as location data, medical history, daily conversations, personal preferences, and behaviour. This information may be misused or shared without a proper consent ~\autocite{Rahman2023}.

The software generally known today as artificial intelligence is mostly cloud-based. Online applications such as ChatGPT generate their responses on external servers and communicate with the user over the internet. Application based software normally uses more of your own device's computing power but still communicates with servers and databases over a network. Both are incapable of providing their assistance without internet traffic, resulting in a permanent risk of your data getting leaked. Until recently, there was no other option available. With Apple's recent WWDC event, which announces their new products and features for the upcoming year, however, they might have just found the solution to this problem. Their new software, called Apple AI, fixes not only the demand for personalisation, but also ensures optimal data privacy. The way Apple is able to achieve this, is by integrating artificial intelligence software into a mobile device's operating system,  iOS or MacOS in this case. This allows the software to have direct access to all of your personal data, providing the most custom tailored assistance, while ensuring complete privacy.

“It has to be powerful enough to help with the things that matter most to you. It has to be intuitive and easy to use. It has to be deeply integrated into your product experiences. Most importantly, it has to understand you and be grounded in your personal context like your routine, your relationships, your communications and more. And of course, it has to be built with privacy from the ground up. Together, this goes beyond artificial intelligence. It's personal intelligence” ~\autocite{Cook2024}.

On paper, this new software could resolve almost every problem encountered today with modern AI. However, because Apple Intelligence was only announced recently and is currently only available in beta versions, the value or use cases for people with ADHD have not been tested yet. If Apple does live up to its promises, this software could become a massive improvement for the day to day lives of people with ADHD, when compared to current existing software. 

% Use the following commands to cite references:
% \autocite{BIBTEXKEY} -> (Author, year)
% \textcite{BIBTEXKEY} -> Author (year)

\section{Methodology}%
\label{sec:methodology}

Because this software is so different and new, we first learn how to use it and how to formulate the best prompts to get the right answer. Then, multiple different use cases are selected to ensure a clear comparison between this new software and current software when executing ADHD assisting tasks. The use cases are then translated into different categories or criteria. 

After that, the real study starts by interviewing groups of all different kinds of people with ADHD, such as students or employed adults. They will be asked about their use of AI in a day to day life. This way, we can compare the different groups of age, sex, interests, profession in the use of AI applications to assist them with the struggles they experience because of their ADHD. They are then asked to download the most recommended app in every category and use this as their main source of assistance. 

After that, we'll ask them to switch to the new Apple Intelligence and now use this as their new personal assistance. In both cases, people are given a detailed explanation about how they can get the most use out of the software. 

In the end, they are asked about their findings and the pros and cons they experienced. They will eventually have to rate both AI models for every category determined at the start.This should normally translate to an obvious answer to our research question.

% Snippet for a figure that can e.g. be used to include a Gantt diagram.
%
% The figure is included in the figure* environment to spread it over both
% columns for better readability. Try not to manipulate the positioning of
% figures (e.g., with [ht!]), but always provide a meaningful caption and label,
% and refer to it in the text.
%
% If you include an image from a source, you must cite the source in the caption
% (command \autocite).
%
% \begin{figure*}
%   \centering
%   \includegraphics[width=\textwidth]{example-image-16x9}
%   \caption{\label{fig:gantt}Gantt diagram of the research phases and milestones.}
% \end{figure*}

\section{Expected results}%
\label{sec:expected-results}


The expectations are for the new Apple Intelligence come out as the ultimate winner and for it to be extremely helpful for people with ADHD. This is because it takes the advantages and use cases from existing software and even improves them while also completely eliminating the current limitations of this sofware.

\section{Discussion, expected conclusion}%
\label{sec:discussion-conclusion}

If the new AI technology truly lives up to its promises, it could potentially revolutionize the way people with ADHD manage their tasks and time.
Having a personal artificial intelligence assistent at the kernel level of your own mobile device would almost completely eliminate the difficulty of setting up a new app and having to manually enter all of your personal information.
By already having access to all relevant data, the new AI software can start assisting the user immediately, giving tailored suggestions and reminders, and automatically set appointments whenever a text or email for example is received, eliminating the need for manual input.
If used to it's full potential, this could greatly improve the life quality of neurodivergent people, as they would no longer have to worry about forgetting important tasks or appointments.

%------------------------------------------------------------------------------
% Referentielijst
%------------------------------------------------------------------------------

\printbibliography[heading=bibintoc]

\end{document}